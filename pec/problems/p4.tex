\textbf{4. Cuestiones}

\vspace{20px}

\begin{enumerate}

    \item Si se eliminan las galaxias con un desplazamiento espectral negativo del cálculo  obtenemos un
    valor de la constante de Hubble $H_0 = 76,9\;km\;s^{-1}\;Mpc^{-1}$.

    Hoy en día, el valor más aceptado, es $70 \pm 5 \; km\;s^{-1}\;Mpc^{-1}$, por lo que nuestro valor calculado
    se encuentra ligeramente fuera del margen de error del valor aceptado actualmente, pero dentro del rango de valores
    históricos que se ha dado a la constante.

    \vspace{20px}

    \item El tiempo Hubble $\tau_0$ o edad del universo, se calcula de la siguiente manera:

    \begin{equation*}
        \tau_0\;(a\tilde{n}os) = \frac{978 \times 10^9}{H_0\; (km\;s^{-1}\;Mpc^{-1})}
        \end{equation*}

    Susituyendo el valor calculado de $H_0$ calculado anteriormente, obtenemos un valor de $1,27 \times 10^{10}$ años; es decir,
    $12,7$ mil millones de años.

    El valor aceptado en la actualidad para la edad del Universo es $13,787 \pm 0,020$ mil millones de años, por lo que de nuevo
    nuestro valor calculado se sitúa fuera del rango aceptado.

    \vspace{20px}

    \item El valor de la constante de Hubble que se obtiene de la gráfica es $72,34 km\;s^{-1}\;Mpc^{-1}$.

    En este caso, sí que se trata de un valor acorde con el valor más aceptado en la actualidad.

    \vspace{20px}

    \item El valor de la edad del Universo que se obtiene a partir de la pendiente de la gráfica es $13,52$ mil millones de años.
    En este caso, el valor calculado se aproxima más que en el apartado 2 al valor más aceptado actualmente, pero todavía se encuentra un orden de magnitud
    por encima del margen de error.

    \vspace{20px}

    \item Las galaxias con $z$ negativo se está aproximando a nosotros.

    Esto no contradice la ley de Hubble, porque tenemos que distinguir entre el desplazamiento al rojo cosmológico, debido a la expansión del universo,
    y el desplazamiento espectral debido al efecto Doppler por el movimiento relativo entre galaxias.

    Debido a la acción de la gravedad, existen galaxias que se están acercando relativamente entre ellas. En esos casos, se produce un desplazamiento
    espectral al azul, que son los valores de $z$ negativos que hemos observado.

    Pero ello no invalida nuestro entendimiento de la expansión del Universo.

    \vspace{20px}

    \item El valor calculado previamente es $7729, 3  \; \si{\angstrom}$.

    Este valor es mayor al calculado en el laboratorio, lo que concuerda con lo esperado, ya que un desplazamiento al rojo de la longitud de onda
    es un incremento en dicha longitud.



\end{enumerate}