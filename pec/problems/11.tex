\question Consideremos la matriz

\begin{equation*}
    A =
    \begin{pmatrix}
        2 & 1 \\
        1 & 1
    \end{pmatrix}
    \in
    \mathcal{M}_2(\mathbb{R})
\end{equation*}

\begin{itemize}[$\bullet$]
    \item Probar que el conjunto $\mathcal{N}$ de matrices que conmutan con A es un subespacio vectorial de $\mathcal{M}_2(\mathbb{R})$.
    \item Calcular su dimensión y una base.
\end{itemize}

\vspace{20px}
\textit{Solución:}

\begin{itemize}[$\bullet$]
    \item El conjunto de matrices que conmutan con $A$ está formado por todas las matrices $B$ que cumplen $AB = BA$.

    Para que los dos productos tengan sentido, $B$ debe ser una matriz cuadrada de orden 2.

    $\mathcal{N}$ es un subespacio vectorial de $\mathcal{M}_2(\mathbb{R})$ si
    para cuales quiera $C, D \in \mathcal{N}$ y $\alpha, \beta \in \mathbb{R}$, se cumple la propiedad:

    \begin{equation*}
        \alpha C + \beta D \in \mathcal{N}
    \end{equation*}

    Lo demostramos de la siguiente manera:

    \begin{equation*}
        A (\alpha C + \beta D) = A (\alpha C) + A (\beta D) = \alpha (AC) + \beta (AD) = \alpha (CA) + \beta (DA) = (\alpha C + \beta D)A,
    \end{equation*}

    donde se ha hecho uso de la propiedad distributiva de la suma de matrices y de que $C,D \in \mathcal{N}$ y por lo tanto conmutan con $A$.

%    por lo que $\mathcal{N}$ es un subconjunto vectorial de  $\mathcal{M}_2(\mathbb{R})$.

    \vspace{20px}

    \item Definimos $B$ como:

    \begin{equation*}
        B =
        \begin{pmatrix}
            a & b \\
            c & d
        \end{pmatrix}
    \end{equation*}

    Tenemos entonces:

    \begin{equation*}
        \begin{pmatrix}
            2 & 1 \\
            1 & 1
        \end{pmatrix}
        \begin{pmatrix}
            a & b \\
            c & d
        \end{pmatrix}
        =
        \begin{pmatrix}
            a & b \\
            c & d
        \end{pmatrix}
        \begin{pmatrix}
            2 & 1 \\
            1 & 1
        \end{pmatrix}
        \Longrightarrow
        \begin{pmatrix}
            2a+c & 2b+d \\
            a+c  & b+d
        \end{pmatrix}
        =
        \begin{pmatrix}
            2a+b & a+b \\
            2c+d & c+d
        \end{pmatrix}
    \end{equation*}

    De la igualdad anterior obtenemos las siguientes ecuaciones:

    \[
        \systeme*{2a + c = 2a + b, 2b + d = a + b, a + c = 2c + d, b + d = c + d}
    \]

    Reordenando:

    \[
        \systeme[abcd]{b - c = 0, a - b + d = 0, a - c - d = 0, b - c = 0}
    \]

    Resolvemos el sistema escalonando:

    \begin{equation*}
        \begin{pmatrix}
            0 & 1  & -1 & 0  \\
            1 & -1 & 0  & 1  \\
            1 & 0  & -1 & -1 \\
            0 & 1  & -1 & 0
        \end{pmatrix}
        \sim
        \begin{pmatrix}
            1 & -1 & 0  & 1  \\
            0 & 1  & -1 & 0  \\
            1 & 0  & -1 & -1 \\
            0 & 1  & -1 & 0
        \end{pmatrix}
        \sim
        \begin{pmatrix}
            1 & -1 & 0  & 1  \\
            0 & 1  & -1 & 0  \\
            0 & 1  & -1 & -2 \\
            0 & 1  & -1 & 0
        \end{pmatrix}
        \sim
        \begin{pmatrix}
            1 & -1 & 0  & 1  \\
            0 & 1  & -1 & 0  \\
            0 & 0  & 0  & -2 \\
            0 & 0  & 0  & 0
        \end{pmatrix}
    \end{equation*}

    La solución del sistema anterior es:

    \begin{equation*}
    (a,b,c,d)
        = (\alpha, \alpha, \alpha, 0) \hspace{12pt} \text{con }\;  \alpha \in \mathbb{R}
    \end{equation*}

    Al depender solo de un parámetro, llegamos a la conclusión de que $\dim(\mathcal{N}) = 1$.\\

    Dando valor a $\alpha = 1$, obtenemos la base $\mathcal{B} = \{ \begin{pmatrix}
                                                                        1 & 1 \\
                                                                        1 & 0
    \end{pmatrix}\}$.

    Comprobamos efectivamente que para cualquier $\alpha \in \mathbb{R}, B = \begin{pmatrix}
                                                                                 \alpha & \alpha \\
                                                                                 \alpha & 0
    \end{pmatrix} $ cumple $AB = BA$:

    \begin{equation*}
        \begin{pmatrix}
            2 & 1 \\
            1 & 1
        \end{pmatrix}
        \begin{pmatrix}
            \alpha & \alpha \\
            \alpha & 0
        \end{pmatrix}
        =
        \begin{pmatrix}
            3\alpha & 2\alpha \\
            2\alpha & \alpha
        \end{pmatrix}
    \end{equation*}

    \begin{equation*}
        \begin{pmatrix}
            \alpha & \alpha \\
            \alpha & 0
        \end{pmatrix}
        \begin{pmatrix}
            2 & 1 \\
            1 & 1
        \end{pmatrix}
        =
        \begin{pmatrix}
            3\alpha & 2\alpha \\
            2\alpha & \alpha
        \end{pmatrix}
    \end{equation*}

\end{itemize}

