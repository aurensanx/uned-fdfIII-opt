\textbf{3. Longitud de onda la línea H$_{\alpha}$ del hidrógeno para la galaxia
con $z$ mayor}

\vspace{20px}

La galaxia con $z$ mayor es LEDA 2816758, con $z = 0,1778$.

Sabemos que el desplazamiento al rojo de las líneas del espectro de las galaxias distantes se define como:

\begin{equation*}
    z = \frac{\Delta \lambda}{\lambda_0} = \frac{(\lambda - \lambda_0)}{\lambda_0}
\end{equation*}

Como se nos dice que $\lambda_0 = 6562,8\;\si{\angstrom}$, la longitud de onda observada de la línea H$_{\alpha}$ para esta galaxia
se puede calcular como:

\begin{equation*}
    \lambda = z \; \lambda_0 + \lambda_0 =  (z + 1)\; \lambda_0
\end{equation*}

Sustituyendo valores:

\begin{equation*}
    \lambda =   (0,1778 + 1) \; 6562,8\;\si{\angstrom} = 7729,3 \; \si{\angstrom}
\end{equation*}
\vspace{20px}
