\textbf{Problema 3.} (2.5 puntos) Sea el problema de valores iniciales para la función $x(t)$ dado por

\begin{equation*}
    \frac{dx}{dt} = -x^2 + 2x+3, \hspace{20pt} x(0) = 0.
\end{equation*}



\begin{enumerate}
[label=(\alph*)]
    \item (0.5 puntos) Justificar por qué el problema tiene solución única en un entorno de $t = 0$.
    \item (2 puntos) Sin calcular explícitamente la solución del problema, hacer razonadamente un
    esquema aproximado de la gráfica de dicha solución.
\end{enumerate}
