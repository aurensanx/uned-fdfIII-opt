\question
\begin{itemize}[$\bullet$]
    \item Sea  $A \in \mathcal{M}_n(\mathbb{R})$ una matriz nilpotente. Calculando explícitamente su inversa, probar que $I_n - A$ es invertible.
    \item Consideremos el espacio vectorial $\mathbb{R}_n[x]$ de los polinomios de grado menor o igual que $n$. Se definen los endomorfismos siguientes.
    \begin{flalign*}
        &D\hspace{8pt}:\hspace{8pt} \mathbb{R}_n[x] \longrightarrow \mathbb{R}_n[x] : P(x) \mapsto P'(x) \\
        &F\hspace{8pt}:\hspace{8pt} \mathbb{R}_n[x] \longrightarrow \mathbb{R}_n[x] : P(x) \mapsto P(x) - P'(x) = (Id - D)(P(x)) \\
    \end{flalign*}
    Razonar que existe el endomorfismo inverso $F^{-1}$ y calcularlo en función del endomorfismo derivada $D$.
\end{itemize}

Indicación: en un entorno del origen de la recta real $\mathbb(R)$, se tiene el siguiente desarrollo en serie de MacLaurin
\begin{equation*}
    \dfrac{1}{1-x} = 1 + x + x^2 + x^3 + \ldots
\end{equation*}

\vspace{20px}
\textit{Solución:}

\begin{itemize}[$\bullet$]
    \item Para probar que $I_n - A$ es invertible debemos encontrar una matriz $A^{-1}$ tal que $AA^{-1} = I_n$.

    Si $A^n = 0$ para algún $n > 0$, por ser nilpotente, podemos escribir $A^{-1}$ como $(I_n + A + A^2 + \ldots + A^{n-1})$.


    \begin{align*}
    (I_n - A)(I_n + A + A^2 + \ldots + A^{n-1})
        \\
        &=I_n - A + A - A^2 + A^2 - A^3 + \ldots + A^{n-1} - A^n\\
        &= I_n - A^n \\
        &= I_n
    \end{align*}

    Así hemos demostrado que $ A^{-1} = (I_n + A + A^2 + \ldots + A^{n-1})$ es la matriz inversa de cualquier matriz nilpotente $A \in \mathcal{M}_n(\mathbb{R})$.

    \vspace{20px}
    \item

    Primero escribimos la matriz del endomorfismo $D$ respecto a la base canónica de $\mathbb{R}_n[x]$, que es una matriz cuadrada de orden $n +1$.

    \begin{equation*}
        \mathcal{M}(D) =
        \begin{pmatrix}
            0      & 1      & 0      & \ldots & 0      \\
            0      & 0      & 2      & \ldots & 0      \\
            \vdots & \vdots & \vdots & \ddots & \vdots \\
            0      & 0      & 0      & \ldots & n      \\
            0      & 0      & 0      & \ldots & 0
        \end{pmatrix}

    \end{equation*}

    A continuación, calculamos la matriz del endomorfismo $F$ a partir del endomorfismo identidad y del endomorfismo $D$.

    \begin{equation*}
        \mathcal{M}(F) =
        \begin{pmatrix}
            1      & -1     & 0      & \ldots & 0      \\
            0      & 1      & -2     & \ldots & 0      \\
            \vdots & \vdots & \vdots & \ddots & \vdots \\
            0      & 0      & 0      & \ldots & -n     \\
            0      & 0      & 0      & \ldots & 1
        \end{pmatrix}
    \end{equation*}

    Existe el endomorfismo inverso $F^{-1}$ si podemos obtener una matriz $\mathcal{M}(F^{-1})$
    tal que
    \begin{equation*}
        \mathcal{M}(F^{-1})\mathcal{M}(F) = I_{n}
    \end{equation*}
    O lo que es lo mismo, si existe $\mathcal{M}^{-1}(F)$.\\

    Como $\mathcal{M}(F)$ es una matriz triangular, su determinante consiste en el producto de los elementos de su traza, y observamos que es distinto de cero,
    ya que  $\det(\mathcal{M}(F)) = 1^n = 1$.
    Por tanto, $rg(\mathcal{M}(F)) = n$ y existe su inversa.\\

    Para dar este endomorfismo en función del endomorfismo derivada $D$, utilizamos el desarrollo en serie de MacLaurin proporcionado de la siguiente manera:

    \begin{equation*}
        F^{-1} (Id - D) = Id\hspace{12pt} \Rightarrow \hspace{12pt}  F^{-1} = (Id - D)^{-1} = Id + D + D^2 + D^3 + \ldots

    \end{equation*}

    Como dato adicional, podemos decir que $\mathcal{M}(D)$ es nilpotente para $k > n$, siendo $n + 1$ la dimensión del espacio vectorial $\mathbb{R}_n[x]$.
    Por lo tanto, el desarrollo en serie empleado anteriormente tiene $n$ elementos dependientes de $D$.


\end{itemize}