\textbf{1. Elaboración de tabla con todos los datos}

\vspace{20px}

Como punto de partida, se recogen los datos de desplazamiento espectral $z$, distancia $r$ (Mpc) y Magnitud aparente visual $m_v$ del software
Stellarium.

La Magnitud absoluta visual $M_v$ se calcula a partir de la Magnitud aparente visual gracias a la relación:

\begin{equation*}
    m - M = 5 \log\frac{r[\text{pc}]}{10} = 5 \log r[\text{pc}] - 5,
\end{equation*}

que equivale a desplazar la galaxia a una distancia estándar de $r$ = 10 pc y medir allí su brillo.

La velocidad de recesión $v$ y la constante de Hubble $H_0$ se obtienen de la relación:

\begin{equation*}
    v = c\,z = H_0\,r,
    \end{equation*}

que nos indica que el desplazamiento espectral $z$ se incrementa con la distancia $r$ de la galaxia, y que existe una relación lineal entre su
distancia y velocidad de recesión.\\

Para hacer los cálculos de las magnitudes mencionadas, se obvian los márgenes de error que nos proporciona Stellarium, aunque sí que se que han
recogido en la tabla pedida.

Los valores numéricos se muestran con delimitadores en notación inglesa.

\begin{table}[H]
    \scriptsize
    \begin{tabular}{|p{64px}|p{80px}|p{57px}|p{54px}|p{54px}|p{54px}|p{60px}|}
        \hline
        \textbf{Identificador de la galaxia} & \textbf{Desplazamiento espectral $z$} & \textbf{Distancia $r$ (Mpc)} & \textbf{Velocidad de recesión $v$ (km/s)}
        & \textbf{Magnitud aparente visual $m_v$} & \textbf{Magnitud absoluta visual $M_v$} & \textbf{Constante de Hubble $H_0$ (km/s Mpc)} \\
        \hline
        \csvreader[late after line=\\]{files/data.csv}{}% use head of csv as column names
        {\csvcoli&\csvcolii&\csvcoliii&\csvcoliv&\csvcolv&\csvcolvi&\csvcolvii}% specify your columns here
        \hline
    \end{tabular}
\end{table}



